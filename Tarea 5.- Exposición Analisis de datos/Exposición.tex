\documentclass[]{article}
\usepackage{lmodern}
\usepackage{amssymb,amsmath}
\usepackage{ifxetex,ifluatex}
\usepackage{fixltx2e} % provides \textsubscript
\ifnum 0\ifxetex 1\fi\ifluatex 1\fi=0 % if pdftex
  \usepackage[T1]{fontenc}
  \usepackage[utf8]{inputenc}
\else % if luatex or xelatex
  \ifxetex
    \usepackage{mathspec}
  \else
    \usepackage{fontspec}
  \fi
  \defaultfontfeatures{Ligatures=TeX,Scale=MatchLowercase}
\fi
% use upquote if available, for straight quotes in verbatim environments
\IfFileExists{upquote.sty}{\usepackage{upquote}}{}
% use microtype if available
\IfFileExists{microtype.sty}{%
\usepackage{microtype}
\UseMicrotypeSet[protrusion]{basicmath} % disable protrusion for tt fonts
}{}
\usepackage[margin=1in]{geometry}
\usepackage{hyperref}
\hypersetup{unicode=true,
            pdfborder={0 0 0},
            breaklinks=true}
\urlstyle{same}  % don't use monospace font for urls
\usepackage{color}
\usepackage{fancyvrb}
\newcommand{\VerbBar}{|}
\newcommand{\VERB}{\Verb[commandchars=\\\{\}]}
\DefineVerbatimEnvironment{Highlighting}{Verbatim}{commandchars=\\\{\}}
% Add ',fontsize=\small' for more characters per line
\usepackage{framed}
\definecolor{shadecolor}{RGB}{248,248,248}
\newenvironment{Shaded}{\begin{snugshade}}{\end{snugshade}}
\newcommand{\AlertTok}[1]{\textcolor[rgb]{0.94,0.16,0.16}{#1}}
\newcommand{\AnnotationTok}[1]{\textcolor[rgb]{0.56,0.35,0.01}{\textbf{\textit{#1}}}}
\newcommand{\AttributeTok}[1]{\textcolor[rgb]{0.77,0.63,0.00}{#1}}
\newcommand{\BaseNTok}[1]{\textcolor[rgb]{0.00,0.00,0.81}{#1}}
\newcommand{\BuiltInTok}[1]{#1}
\newcommand{\CharTok}[1]{\textcolor[rgb]{0.31,0.60,0.02}{#1}}
\newcommand{\CommentTok}[1]{\textcolor[rgb]{0.56,0.35,0.01}{\textit{#1}}}
\newcommand{\CommentVarTok}[1]{\textcolor[rgb]{0.56,0.35,0.01}{\textbf{\textit{#1}}}}
\newcommand{\ConstantTok}[1]{\textcolor[rgb]{0.00,0.00,0.00}{#1}}
\newcommand{\ControlFlowTok}[1]{\textcolor[rgb]{0.13,0.29,0.53}{\textbf{#1}}}
\newcommand{\DataTypeTok}[1]{\textcolor[rgb]{0.13,0.29,0.53}{#1}}
\newcommand{\DecValTok}[1]{\textcolor[rgb]{0.00,0.00,0.81}{#1}}
\newcommand{\DocumentationTok}[1]{\textcolor[rgb]{0.56,0.35,0.01}{\textbf{\textit{#1}}}}
\newcommand{\ErrorTok}[1]{\textcolor[rgb]{0.64,0.00,0.00}{\textbf{#1}}}
\newcommand{\ExtensionTok}[1]{#1}
\newcommand{\FloatTok}[1]{\textcolor[rgb]{0.00,0.00,0.81}{#1}}
\newcommand{\FunctionTok}[1]{\textcolor[rgb]{0.00,0.00,0.00}{#1}}
\newcommand{\ImportTok}[1]{#1}
\newcommand{\InformationTok}[1]{\textcolor[rgb]{0.56,0.35,0.01}{\textbf{\textit{#1}}}}
\newcommand{\KeywordTok}[1]{\textcolor[rgb]{0.13,0.29,0.53}{\textbf{#1}}}
\newcommand{\NormalTok}[1]{#1}
\newcommand{\OperatorTok}[1]{\textcolor[rgb]{0.81,0.36,0.00}{\textbf{#1}}}
\newcommand{\OtherTok}[1]{\textcolor[rgb]{0.56,0.35,0.01}{#1}}
\newcommand{\PreprocessorTok}[1]{\textcolor[rgb]{0.56,0.35,0.01}{\textit{#1}}}
\newcommand{\RegionMarkerTok}[1]{#1}
\newcommand{\SpecialCharTok}[1]{\textcolor[rgb]{0.00,0.00,0.00}{#1}}
\newcommand{\SpecialStringTok}[1]{\textcolor[rgb]{0.31,0.60,0.02}{#1}}
\newcommand{\StringTok}[1]{\textcolor[rgb]{0.31,0.60,0.02}{#1}}
\newcommand{\VariableTok}[1]{\textcolor[rgb]{0.00,0.00,0.00}{#1}}
\newcommand{\VerbatimStringTok}[1]{\textcolor[rgb]{0.31,0.60,0.02}{#1}}
\newcommand{\WarningTok}[1]{\textcolor[rgb]{0.56,0.35,0.01}{\textbf{\textit{#1}}}}
\usepackage{graphicx,grffile}
\makeatletter
\def\maxwidth{\ifdim\Gin@nat@width>\linewidth\linewidth\else\Gin@nat@width\fi}
\def\maxheight{\ifdim\Gin@nat@height>\textheight\textheight\else\Gin@nat@height\fi}
\makeatother
% Scale images if necessary, so that they will not overflow the page
% margins by default, and it is still possible to overwrite the defaults
% using explicit options in \includegraphics[width, height, ...]{}
\setkeys{Gin}{width=\maxwidth,height=\maxheight,keepaspectratio}
\IfFileExists{parskip.sty}{%
\usepackage{parskip}
}{% else
\setlength{\parindent}{0pt}
\setlength{\parskip}{6pt plus 2pt minus 1pt}
}
\setlength{\emergencystretch}{3em}  % prevent overfull lines
\providecommand{\tightlist}{%
  \setlength{\itemsep}{0pt}\setlength{\parskip}{0pt}}
\setcounter{secnumdepth}{0}
% Redefines (sub)paragraphs to behave more like sections
\ifx\paragraph\undefined\else
\let\oldparagraph\paragraph
\renewcommand{\paragraph}[1]{\oldparagraph{#1}\mbox{}}
\fi
\ifx\subparagraph\undefined\else
\let\oldsubparagraph\subparagraph
\renewcommand{\subparagraph}[1]{\oldsubparagraph{#1}\mbox{}}
\fi

%%% Use protect on footnotes to avoid problems with footnotes in titles
\let\rmarkdownfootnote\footnote%
\def\footnote{\protect\rmarkdownfootnote}

%%% Change title format to be more compact
\usepackage{titling}

% Create subtitle command for use in maketitle
\providecommand{\subtitle}[1]{
  \posttitle{
    \begin{center}\large#1\end{center}
    }
}

\setlength{\droptitle}{-2em}

  \title{}
    \pretitle{\vspace{\droptitle}}
  \posttitle{}
    \author{}
    \preauthor{}\postauthor{}
    \date{}
    \predate{}\postdate{}
  

\begin{document}

\hypertarget{trabajando-con-datasets-externos}{%
\section{Trabajando con Datasets
Externos}\label{trabajando-con-datasets-externos}}

Manejo de datasets externos para saber las ventajas que hay al trabajar
con estos hecho de manera manual o utilizando otros datasets ya
formulados. Dos de los formatos que puede manejar Rstudio son: 1.
HTML(Hypertext Markup Language) 2. CSV(Comma Separated Value)

Un ejemplo de uso de un dataset creado manualmente usando una imagen del
precio del Grande Latte de Starbucks
\url{http://www.coventryleague.com/blogentary/the-starbucks-latte-index}

\begin{Shaded}
\begin{Highlighting}[]
\KeywordTok{library}\NormalTok{(readr)}
\KeywordTok{library}\NormalTok{(rpart)}
\NormalTok{LatteIndexFrame <-}\StringTok{ }\KeywordTok{read_csv}\NormalTok{(}\StringTok{"LatteIndexFrame.csv"}\NormalTok{)}
\end{Highlighting}
\end{Shaded}

\begin{verbatim}
## Parsed with column specification:
## cols(
##   city = col_character(),
##   price = col_double(),
##   country = col_character()
## )
\end{verbatim}

\begin{Shaded}
\begin{Highlighting}[]
\NormalTok{Mids <-}\StringTok{ }\KeywordTok{barplot}\NormalTok{(LatteIndexFrame}\OperatorTok{$}\NormalTok{price, }\DataTypeTok{col=}\StringTok{"transparent"}\NormalTok{,}\DataTypeTok{horiz =} \OtherTok{TRUE}\NormalTok{, }\DataTypeTok{xlim =} \KeywordTok{c}\NormalTok{(}\DecValTok{0}\NormalTok{,}\DecValTok{11}\NormalTok{))}
\KeywordTok{text}\NormalTok{(}\FloatTok{8.5}\NormalTok{,Mids,LatteIndexFrame}\OperatorTok{$}\NormalTok{city,}\DataTypeTok{cex =} \FloatTok{0.8}\NormalTok{)}
\KeywordTok{text}\NormalTok{(}\FloatTok{10.5}\NormalTok{,Mids,}\KeywordTok{format}\NormalTok{(LatteIndexFrame}\OperatorTok{$}\NormalTok{price,}\DataTypeTok{nsmall=}\DecValTok{2}\NormalTok{), }\DataTypeTok{cex =} \FloatTok{0.8}\NormalTok{)}
\end{Highlighting}
\end{Shaded}

\includegraphics{Exposición_files/figure-html/unnamed-chunk-1-1.png}

\hypertarget{administraciuxf3n-de-archivos-en-r}{%
\subsection{Administración de archivos en
R}\label{administraciuxf3n-de-archivos-en-r}}

Para poder usar un archivo el programa tiene que encontrarse en el
directorio en donde este se encuentra, a veces no solo es necesario eso
si no que hay que crear o movernos entre directorios para eso estan los
siguientes comandos.

\begin{Shaded}
\begin{Highlighting}[]
\KeywordTok{getwd}\NormalTok{() }\CommentTok{#Obtenemos la dirección actual donde se encuentra el programa.}
\end{Highlighting}
\end{Shaded}

\begin{verbatim}
## [1] "/home/jose/Escritorio/Maestria/Tercer Cuatrimestre/Said/Tarea 6.- Exposición Analisis de datos"
\end{verbatim}

\begin{Shaded}
\begin{Highlighting}[]
\KeywordTok{setwd}\NormalTok{(}\StringTok{".."}\NormalTok{) }\CommentTok{#Para movernos a una carpeta dentro de nuestro directorio o atras}
\KeywordTok{getwd}\NormalTok{()}
\end{Highlighting}
\end{Shaded}

\begin{verbatim}
## [1] "/home/jose/Escritorio/Maestria/Tercer Cuatrimestre/Said"
\end{verbatim}

\begin{Shaded}
\begin{Highlighting}[]
\KeywordTok{setwd}\NormalTok{(}\StringTok{"../Tarea 6.- Exposición Analisis de datos"}\NormalTok{)}
\KeywordTok{getwd}\NormalTok{()}
\end{Highlighting}
\end{Shaded}

\begin{verbatim}
## [1] "/home/jose/Escritorio/Maestria/Tercer Cuatrimestre/Said/Tarea 6.- Exposición Analisis de datos"
\end{verbatim}

\begin{Shaded}
\begin{Highlighting}[]
\KeywordTok{list.files}\NormalTok{() }\CommentTok{#Para obtener un listado de los archivos encontrados en el directorio actual.}
\end{Highlighting}
\end{Shaded}

\begin{verbatim}
##  [1] "(Chapman & Hall_CRC Data Mining and Knowledge Discovery Series) Ronald K. Pearson - Exploratory Data Analysis Using R-Chapman and Hall_CRC (2018).pdf"
##  [2] "auto-mpg.names"                                                                                                                                       
##  [3] "AutoMpgBoxplotEx.pdf"                                                                                                                                 
##  [4] "AutoMpgBoxplotEx.png"                                                                                                                                 
##  [5] "BMfile2000-Jul2015.xls"                                                                                                                               
##  [6] "Dataset.csv"                                                                                                                                          
##  [7] "Exposición_files"                                                                                                                                     
##  [8] "Exposición.html"                                                                                                                                      
##  [9] "Exposición.log"                                                                                                                                       
## [10] "Exposición.Rmd"                                                                                                                                       
## [11] "Exposición.tex"                                                                                                                                       
## [12] "holamundo.csv"                                                                                                                                        
## [13] "LatteIndexFrame.csv"                                                                                                                                  
## [14] "linearModelExample.rds"                                                                                                                               
## [15] "UCIautoMpg.txt"                                                                                                                                       
## [16] "Unclaimed_Bank_Accounts-2.csv"                                                                                                                        
## [17] "Untitled.ipynb"                                                                                                                                       
## [18] "WBCDDataset.csv"
\end{verbatim}

\begin{Shaded}
\begin{Highlighting}[]
\KeywordTok{list.files}\NormalTok{(}\DataTypeTok{pattern =} \StringTok{"Expo"}\NormalTok{) }\CommentTok{#Obtener un listado de los archivos con caracteres comunes.}
\end{Highlighting}
\end{Shaded}

\begin{verbatim}
## [1] "Exposición_files" "Exposición.html"  "Exposición.log"  
## [4] "Exposición.Rmd"   "Exposición.tex"
\end{verbatim}

\hypertarget{colocar-datos-manualmente}{%
\subsection{Colocar datos Manualmente}\label{colocar-datos-manualmente}}

\hypertarget{colocar-datos-a-mano}{%
\subsubsection{Colocar datos a mano}\label{colocar-datos-a-mano}}

Esta es una de las formas de hacer nuestro propio dataset, debido a que
es una forma sencilla de trabajar y colocar datos como se hizo con el
dataset del precio del latte grande de starbucks y puede servir como
guía para empezar a trabajar.

\hypertarget{es-malo-colocar-los-datos-pero-a-veces-hay-excepciones}{%
\subsubsection{Es malo colocar los datos pero a veces hay
excepciones}\label{es-malo-colocar-los-datos-pero-a-veces-hay-excepciones}}

\begin{enumerate}
\def\labelenumi{\arabic{enumi}.}
\tightlist
\item
  Es muy difícil colocar datos cuando es un conjunto muy largo de estos.
\item
  Podemos cometer un error al colocar los datos el cual cause que no
  sirva.
\end{enumerate}

\hypertarget{interactuando-con-el-internet}{%
\subsection{Interactuando con el
Internet}\label{interactuando-con-el-internet}}

El internet nos ayuda a encontrar dataset para probar algunas metodos en
nuestro programa en R, si bien R ya tiene un conjunto de dataset para
empezar a programar es bueno saber como introducir nuevos datasets y
trabajar con ellos.

\hypertarget{previa-vista-de-3-ejemplos-de-datasets-en-internet}{%
\subsubsection{Previa vista de 3 ejemplos de datasets en
Internet}\label{previa-vista-de-3-ejemplos-de-datasets-en-internet}}

\begin{itemize}
\tightlist
\item
  Distancia recorrida en millas de un automovil.
\end{itemize}

\begin{Shaded}
\begin{Highlighting}[]
\KeywordTok{browseURL}\NormalTok{(}\StringTok{"http://archive.ics.uci.edu/ml"}\NormalTok{) }\CommentTok{#Nos abrirá una pestaña en el navegador a la dirección entre comillas.}
\NormalTok{URL <-}\StringTok{ "http://archive.ics.uci.edu/ml/machine-learning-databases/auto-mpg/auto-mpg.data"}
\KeywordTok{download.file}\NormalTok{(URL, }\StringTok{"UCIautoMpg.txt"}\NormalTok{) }\CommentTok{#Nos descargará el dataset seleccionado}
\end{Highlighting}
\end{Shaded}

\begin{itemize}
\tightlist
\item
  Cuentas de bancos no reclamadas en Canada
\end{itemize}

\begin{Shaded}
\begin{Highlighting}[]
\NormalTok{SocrataURL <-}\StringTok{ "https://opendata.socrata.com/"}
\KeywordTok{browseURL}\NormalTok{(SocrataURL)}
\NormalTok{DownloadFile <-}\StringTok{ "C:/Users/Ron/Downloads/Unclaimed_bank_accounts.csv"}
\KeywordTok{file.copy}\NormalTok{(DownloadFile, }\StringTok{"Unclaimed_bank_accounts.csv"}\NormalTok{)}
\end{Highlighting}
\end{Shaded}

\begin{verbatim}
## [1] FALSE
\end{verbatim}

\begin{itemize}
\tightlist
\item
  Indice de economia de la BigMac
\end{itemize}

\begin{Shaded}
\begin{Highlighting}[]
\NormalTok{Bigmac <-}\StringTok{ "http://www.economist.com/content/big-mac-index"}
\KeywordTok{browseURL}\NormalTok{(Bigmac)}
\end{Highlighting}
\end{Shaded}

\hypertarget{trabajando-con-archivos-csv}{%
\subsection{Trabajando con Archivos
CSV}\label{trabajando-con-archivos-csv}}

Los archivos CSV son uno de los mas usados para la lectura de datos
debido a su simplicidad al manejarlos y el gran numero de programas que
pueden utilizarlos.

\hypertarget{leer-y-escribir-en-un-archivos-csv}{%
\subsubsection{Leer y escribir en un archivos
CSV}\label{leer-y-escribir-en-un-archivos-csv}}

La forma mas sencilla de leer y escribir en un archivos CSV es usando
las funciones: 1. read.csv(``Nombre del archivo CSV'')

\begin{Shaded}
\begin{Highlighting}[]
\NormalTok{WBCD <-}\StringTok{ }\KeywordTok{read.csv}\NormalTok{(}\StringTok{"Dataset.csv"}\NormalTok{)}
\KeywordTok{head}\NormalTok{(WBCD)}
\end{Highlighting}
\end{Shaded}

\begin{verbatim}
##         ID MB Radio Textura Perimetro   Area Smoothness Compactness
## 1   842302  M 17.99   10.38    122.80 1001.0    0.11840     0.27760
## 2   842517  M 20.57   17.77    132.90 1326.0    0.08474     0.07864
## 3 84300903  M 19.69   21.25    130.00 1203.0    0.10960     0.15990
## 4 84348301  M 11.42   20.38     77.58  386.1    0.14250     0.28390
## 5 84358402  M 20.29   14.34    135.10 1297.0    0.10030     0.13280
## 6   843786  M 12.45   15.70     82.57  477.1    0.12780     0.17000
##   Concavity Concave.Points Symmetry Fractal.Dimention Radius.Error
## 1    0.3001        0.14710   0.2419           0.07871       1.0950
## 2    0.0869        0.07017   0.1812           0.05667       0.5435
## 3    0.1974        0.12790   0.2069           0.05999       0.7456
## 4    0.2414        0.10520   0.2597           0.09744       0.4956
## 5    0.1980        0.10430   0.1809           0.05883       0.7572
## 6    0.1578        0.08089   0.2087           0.07613       0.3345
##   Texture.Error Perimeter.Error AreasError Smoothness.Error
## 1        0.9053           8.589     153.40         0.006399
## 2        0.7339           3.398      74.08         0.005225
## 3        0.7869           4.585      94.03         0.006150
## 4        1.1560           3.445      27.23         0.009110
## 5        0.7813           5.438      94.44         0.011490
## 6        0.8902           2.217      27.19         0.007510
##   Compactness.Error Concavity.Error Concave.Points.Error Symmetry.Error
## 1           0.04904         0.05373              0.01587        0.03003
## 2           0.01308         0.01860              0.01340        0.01389
## 3           0.04006         0.03832              0.02058        0.02250
## 4           0.07458         0.05661              0.01867        0.05963
## 5           0.02461         0.05688              0.01885        0.01756
## 6           0.03345         0.03672              0.01137        0.02165
##   Fractal.Dimention.Error Worst.Radius Worst.Texture Worst.Perimeter
## 1                0.006193        25.38         17.33          184.60
## 2                0.003532        24.99         23.41          158.80
## 3                0.004571        23.57         25.53          152.50
## 4                0.009208        14.91         26.50           98.87
## 5                0.005115        22.54         16.67          152.20
## 6                0.005082        15.47         23.75          103.40
##   Worst.Area Worst.Smoothness Worst.Compactness Worst.Concavity
## 1     2019.0           0.1622            0.6656          0.7119
## 2     1956.0           0.1238            0.1866          0.2416
## 3     1709.0           0.1444            0.4245          0.4504
## 4      567.7           0.2098            0.8663          0.6869
## 5     1575.0           0.1374            0.2050          0.4000
## 6      741.6           0.1791            0.5249          0.5355
##   Worst.Concave.Points Worst.Symmetry Worst.Fractal.Dimension
## 1               0.2654         0.4601                 0.11890
## 2               0.1860         0.2750                 0.08902
## 3               0.2430         0.3613                 0.08758
## 4               0.2575         0.6638                 0.17300
## 5               0.1625         0.2364                 0.07678
## 6               0.1741         0.3985                 0.12440
\end{verbatim}

\begin{enumerate}
\def\labelenumi{\arabic{enumi}.}
\setcounter{enumi}{1}
\tightlist
\item
  write.csv(``Los datos'', ``El nombre del archivo CSV'')
\end{enumerate}

\begin{Shaded}
\begin{Highlighting}[]
\KeywordTok{write.csv}\NormalTok{(WBCD,}\StringTok{"WBCDDataset.csv"}\NormalTok{)}
\end{Highlighting}
\end{Shaded}

\hypertarget{hojas-de-calculo-vs-archivos-csv-no-son-lo-mismo}{%
\subsubsection{Hojas de calculo vs archivos CSV no son lo
mismo}\label{hojas-de-calculo-vs-archivos-csv-no-son-lo-mismo}}

\begin{enumerate}
\def\labelenumi{\arabic{enumi}.}
\tightlist
\item
  Un archivo CSV es un simple archivo de datos, puede ser leido por
  Microsoft Excel y por muchos otros programas.
\item
  El programa de hoja de calculo de Microsoft Excel es un software que
  puede hacer muchas cosas, incluyendo leer y escribir archivos CSV,
  realizar computación simple de análisis de datos y generar gráficas.
\item
  Un archivo de datos de Microsoft Excel contiene los datos sobre los
  cuales una hoja de calculo se basa.
\end{enumerate}

\begin{Shaded}
\begin{Highlighting}[]
\KeywordTok{library}\NormalTok{(xlsx)}
\NormalTok{BigMacJan2013 <-}\StringTok{ }\KeywordTok{read.xlsx}\NormalTok{(}\StringTok{"BMfile2000-Jul2015.xls"}\NormalTok{, }\DataTypeTok{sheetName =} \StringTok{"Jan2013"}\NormalTok{)}
\KeywordTok{head}\NormalTok{(BigMacJan2013)}
\end{Highlighting}
\end{Shaded}

\begin{verbatim}
##     Country local_price   dollar_ex dollar_price  dollar_ppp
## 1 Argentina       19.00   4.9765000     3.817944   4.3504186
## 2 Australia        4.70   0.9590946     4.900455   1.0761562
## 3    Brazil       11.25   1.9933500     5.643766   2.5759057
## 4   Britain        2.69   0.6332120     4.248183   0.6159277
## 5    Canada        5.41   1.0029000     5.394356   1.2387244
## 6     Chile     2050.00 471.7500000     4.345522 469.3872683
##   dollar_valuation dollar_adj_valuation euro_adj_valuation
## 1       -12.580758            31.583765          9.1424392
## 2        12.205424            -3.442778        -19.9104025
## 3        29.224960            89.815124         57.4425667
## 4        -2.729621             6.361586        -11.7781520
## 5        23.514254            21.154868          0.4921682
## 6        -0.500844            43.103425         18.6974462
##   sterling_adj_valuation yen_adj_valuation yuan_adj_valuation
## 1              23.713617          59.93016         37.3100903
## 2              -9.217955          17.35804          0.7592459
## 3              78.462105         130.70600         98.0755902
## 4               0.000000          29.27450         10.9902803
## 5              13.908482          47.25462         26.4273439
## 6              34.544275          73.93144         49.3310684
\end{verbatim}

\hypertarget{problemas-potenciales-al-usar-archivos-csv-en-r.}{%
\subsubsection{Problemas potenciales al usar archivos CSV en
R.}\label{problemas-potenciales-al-usar-archivos-csv-en-r.}}

\begin{itemize}
\tightlist
\item
  Cambio de nombres Debido a que R solo acepta variables con ``.'' y
  "\_" al momento de querer leer un archivo CSV las variables que
  contengan caracteres no legibles se cambiaran por un punto.
\end{itemize}

\begin{Shaded}
\begin{Highlighting}[]
\KeywordTok{colnames}\NormalTok{(WBCD)}
\end{Highlighting}
\end{Shaded}

\begin{verbatim}
##  [1] "ID"                      "MB"                     
##  [3] "Radio"                   "Textura"                
##  [5] "Perimetro"               "Area"                   
##  [7] "Smoothness"              "Compactness"            
##  [9] "Concavity"               "Concave.Points"         
## [11] "Symmetry"                "Fractal.Dimention"      
## [13] "Radius.Error"            "Texture.Error"          
## [15] "Perimeter.Error"         "AreasError"             
## [17] "Smoothness.Error"        "Compactness.Error"      
## [19] "Concavity.Error"         "Concave.Points.Error"   
## [21] "Symmetry.Error"          "Fractal.Dimention.Error"
## [23] "Worst.Radius"            "Worst.Texture"          
## [25] "Worst.Perimeter"         "Worst.Area"             
## [27] "Worst.Smoothness"        "Worst.Compactness"      
## [29] "Worst.Concavity"         "Worst.Concave.Points"   
## [31] "Worst.Symmetry"          "Worst.Fractal.Dimension"
\end{verbatim}

\begin{itemize}
\tightlist
\item
  Problemas de comas incrustadas Hay ocaciones en que al guardar un
  archivo CSV con algunas comas hace que en veces de separar en 2
  columnas nos agrega todo en una sola y esto hace que ocurra el
  siguiente error.
\end{itemize}

\begin{Shaded}
\begin{Highlighting}[]
\KeywordTok{library}\NormalTok{(sqldf)}
\end{Highlighting}
\end{Shaded}

\begin{verbatim}
## Loading required package: gsubfn
\end{verbatim}

\begin{verbatim}
## Loading required package: proto
\end{verbatim}

\begin{verbatim}
## Loading required package: RSQLite
\end{verbatim}

\begin{Shaded}
\begin{Highlighting}[]
\NormalTok{unclaimedLines <-}\StringTok{ }\KeywordTok{readLines}\NormalTok{(}\StringTok{"Unclaimed_Bank_Accounts-2.csv"}\NormalTok{)}
\NormalTok{line8 <-}\StringTok{ }\NormalTok{unclaimedLines[}\DecValTok{5}\NormalTok{]}
\KeywordTok{strsplit}\NormalTok{(line8, }\DataTypeTok{split =} \StringTok{","}\NormalTok{)}
\end{Highlighting}
\end{Shaded}

\begin{verbatim}
## [[1]]
## [1] "KELM"                   "SELMA       "          
## [3] "67510.34"               "09/08/1995 12:00:00 AM"
## [5] "BANK OF MONTREAL "
\end{verbatim}

\hypertarget{trabajando-con-otro-tipo-de-archivos}{%
\subsection{Trabajando con otro tipo de
archivos}\label{trabajando-con-otro-tipo-de-archivos}}

Los archivos de texto (.txt) son leidos por la funcion ``readLines()'',
esta funcion es usada cuando se quiere leer lineas en especifico de
archivo y los archivos txt son perfectos para esto.

\begin{Shaded}
\begin{Highlighting}[]
\NormalTok{autoMpgRecords <-}\StringTok{ }\KeywordTok{readLines}\NormalTok{(}\StringTok{"UCIautoMpg.txt"}\NormalTok{)}
\NormalTok{x <-}\StringTok{ }\NormalTok{autoMpgRecords[}\DecValTok{1}\NormalTok{]}
\KeywordTok{nchar}\NormalTok{(x)  }\CommentTok{#Numero de caracteres}
\end{Highlighting}
\end{Shaded}

\begin{verbatim}
## [1] 84
\end{verbatim}

\begin{Shaded}
\begin{Highlighting}[]
\KeywordTok{substr}\NormalTok{(x, }\DecValTok{1}\NormalTok{, }\DecValTok{56}\NormalTok{) }\CommentTok{#Limite de comienzo (1) y fin (56) de lectura de caracteres}
\end{Highlighting}
\end{Shaded}

\begin{verbatim}
## [1] "18.0   8   307.0      130.0      3504.      12.0   70  1"
\end{verbatim}

\begin{Shaded}
\begin{Highlighting}[]
\KeywordTok{substr}\NormalTok{(x, }\DecValTok{57}\NormalTok{, }\DecValTok{84}\NormalTok{)}
\end{Highlighting}
\end{Shaded}

\begin{verbatim}
## [1] "\t\"chevrolet chevelle malibu\""
\end{verbatim}

\begin{Shaded}
\begin{Highlighting}[]
\NormalTok{autoMpgNames <-}\StringTok{ }\KeywordTok{readLines}\NormalTok{(}\StringTok{"auto-mpg.names"}\NormalTok{)}
\NormalTok{autoMpgNames[}\DecValTok{32}\OperatorTok{:}\DecValTok{44}\NormalTok{]}
\end{Highlighting}
\end{Shaded}

\begin{verbatim}
##  [1] "7. Attribute Information:"                                    
##  [2] ""                                                             
##  [3] "    1. mpg:           continuous"                             
##  [4] "    2. cylinders:     multi-valued discrete"                  
##  [5] "    3. displacement:  continuous"                             
##  [6] "    4. horsepower:    continuous"                             
##  [7] "    5. weight:        continuous"                             
##  [8] "    6. acceleration:  continuous"                             
##  [9] "    7. model year:    multi-valued discrete"                  
## [10] "    8. origin:        multi-valued discrete"                  
## [11] "    9. car name:      string (unique for each instance)"      
## [12] ""                                                             
## [13] "8. Missing Attribute Values:  horsepower has 6 missing values"
\end{verbatim}

\hypertarget{guardando-y-abriendo-objetos-de-r}{%
\subsubsection{Guardando y Abriendo objetos de
R}\label{guardando-y-abriendo-objetos-de-r}}

El guardado de objetos de R nos ayuda a no perder estos objetos que
podríamos utilizar en otras secciones de R ya que al momento de cerrar
el programa estos objetos desaparecen. Para guardas estas funciones se
utilizan la siguiente función: 1. saveRDS(``objeto'', ``nombre del
archivo.rds'') 2. readRDS(``Nombre del archivo.rds'')

\begin{Shaded}
\begin{Highlighting}[]
\NormalTok{linearModel <-}\StringTok{ }\KeywordTok{lm}\NormalTok{(mpg }\OperatorTok{~}\StringTok{ }\NormalTok{., }\DataTypeTok{data =}\NormalTok{ mtcars)}
\KeywordTok{names}\NormalTok{(linearModel)}
\end{Highlighting}
\end{Shaded}

\begin{verbatim}
##  [1] "coefficients"  "residuals"     "effects"       "rank"         
##  [5] "fitted.values" "assign"        "qr"            "df.residual"  
##  [9] "xlevels"       "call"          "terms"         "model"
\end{verbatim}

\begin{Shaded}
\begin{Highlighting}[]
\KeywordTok{saveRDS}\NormalTok{(linearModel, }\StringTok{"linearModelExample.rds"}\NormalTok{)}
\NormalTok{recoveredLinearModel <-}\StringTok{ }\KeywordTok{readRDS}\NormalTok{(}\StringTok{"linearModelExample.rds"}\NormalTok{)}
\KeywordTok{identical}\NormalTok{(recoveredLinearModel, linearModel)}
\end{Highlighting}
\end{Shaded}

\begin{verbatim}
## [1] TRUE
\end{verbatim}

\hypertarget{archivos-de-gruxe1ficas}{%
\subsubsection{Archivos de Gráficas}\label{archivos-de-gruxe1ficas}}

Función para guardar gráficas a archivos pdf.

\begin{Shaded}
\begin{Highlighting}[]
\KeywordTok{pdf}\NormalTok{(}\StringTok{"AutoMpgBoxplotEx.pdf"}\NormalTok{)}
  \KeywordTok{boxplot}\NormalTok{(Area }\OperatorTok{~}\StringTok{ }\NormalTok{AreasError, }\DataTypeTok{data =}\NormalTok{ WBCD,}
          \DataTypeTok{xlab =} \StringTok{"Area"}\NormalTok{, }\DataTypeTok{ylab =} \StringTok{"AreasError"}\NormalTok{,}
          \DataTypeTok{las =} \DecValTok{1}\NormalTok{, }\DataTypeTok{varwidth =} \OtherTok{TRUE}\NormalTok{)}
\KeywordTok{dev.off}\NormalTok{()}
\end{Highlighting}
\end{Shaded}

\begin{verbatim}
## pdf 
##   2
\end{verbatim}

También se pueden guardar gráficas en forma de imagenes.

\begin{Shaded}
\begin{Highlighting}[]
\KeywordTok{png}\NormalTok{(}\StringTok{"AutoMpgBoxplotEx.png"}\NormalTok{)}
  \KeywordTok{boxplot}\NormalTok{(Area }\OperatorTok{~}\StringTok{ }\NormalTok{AreasError, }\DataTypeTok{data =}\NormalTok{ WBCD,}
          \DataTypeTok{xlab =} \StringTok{"Area"}\NormalTok{, }\DataTypeTok{ylab =} \StringTok{"AreasError"}\NormalTok{,}
          \DataTypeTok{las =} \DecValTok{1}\NormalTok{, }\DataTypeTok{varwidth =} \OtherTok{TRUE}\NormalTok{)}
\KeywordTok{dev.off}\NormalTok{()}
\end{Highlighting}
\end{Shaded}

\begin{verbatim}
## pdf 
##   2
\end{verbatim}

\hypertarget{fucionando-datos-de-diferentes-fuentes}{%
\subsection{Fucionando Datos de diferentes
fuentes}\label{fucionando-datos-de-diferentes-fuentes}}

En ciertas ocaciones es útil juntar un dos dataset diferentes para ver
información de utilidad y para esto se usa la funcion merge(``Primer
dataset'',``Segundo dataset'', by.x = ``Atributo del primero'', by.y =
``Atributo del segundo'')

\begin{Shaded}
\begin{Highlighting}[]
\NormalTok{FlawedMergeFrame <-}\StringTok{ }\KeywordTok{merge}\NormalTok{(LatteIndexFrame, BigMacJan2013, }\DataTypeTok{by.x=}\StringTok{"country"}\NormalTok{, }\DataTypeTok{by.y=}\StringTok{"Country"}\NormalTok{)}
\end{Highlighting}
\end{Shaded}

La forma de ordenar mejor la union es la siguiente:

\begin{Shaded}
\begin{Highlighting}[]
\NormalTok{LatteSubset <-}\StringTok{ }\KeywordTok{data.frame}\NormalTok{(}\DataTypeTok{country =} \KeywordTok{as.character}\NormalTok{(LatteIndexFrame}\OperatorTok{$}\NormalTok{country),}
                          \DataTypeTok{city =} \KeywordTok{as.character}\NormalTok{(LatteIndexFrame}\OperatorTok{$}\NormalTok{city),}
                          \DataTypeTok{GrandeLatteIndex =}\NormalTok{ LatteIndexFrame}\OperatorTok{$}\NormalTok{price,}
                          \DataTypeTok{stringsAsFactors =} \OtherTok{FALSE}\NormalTok{)}
\NormalTok{BigMacSubset <-}\StringTok{ }\KeywordTok{data.frame}\NormalTok{(}\DataTypeTok{country =} \KeywordTok{as.character}\NormalTok{(BigMacJan2013}\OperatorTok{$}\NormalTok{Country),}
                          \DataTypeTok{BigMacIndex =}\NormalTok{ BigMacJan2013}\OperatorTok{$}\NormalTok{dollar_price,}
                          \DataTypeTok{stringsAsFactors =} \OtherTok{FALSE}\NormalTok{)}
\NormalTok{BetterMerge <-}\StringTok{ }\KeywordTok{merge}\NormalTok{(LatteSubset, BigMacSubset)}
\KeywordTok{str}\NormalTok{(BetterMerge, }\DataTypeTok{vec.len =} \DecValTok{2}\NormalTok{)}
\end{Highlighting}
\end{Shaded}

\begin{verbatim}
## 'data.frame':    28 obs. of  4 variables:
##  $ country         : chr  "Argentina" "Australia" ...
##  $ city            : chr  "Buenos Aires" "Sydney" ...
##  $ GrandeLatteIndex: num  4.18 4.82 5.65 4.23 3.87 ...
##  $ BigMacIndex     : num  3.82 4.9 ...
\end{verbatim}

Para observar cuales son las diferencias que hay entre los dos datasets
se usa lo siguiente:

\begin{Shaded}
\begin{Highlighting}[]
\KeywordTok{setdiff}\NormalTok{(LatteSubset}\OperatorTok{$}\NormalTok{country, BetterMerge}\OperatorTok{$}\NormalTok{country)}
\end{Highlighting}
\end{Shaded}

\begin{verbatim}
## [1] "England"
\end{verbatim}

Para cambiar el string usamos el siguiente codigo:

\begin{Shaded}
\begin{Highlighting}[]
\NormalTok{EnglandIndex <-}\StringTok{ }\KeywordTok{which}\NormalTok{(LatteSubset}\OperatorTok{$}\NormalTok{country }\OperatorTok{==}\StringTok{ "England"}\NormalTok{)}
\NormalTok{LatteSubset}\OperatorTok{$}\NormalTok{country[EnglandIndex] <-}\StringTok{ "Britain"}
\NormalTok{FinalMerge <-}\StringTok{ }\KeywordTok{merge}\NormalTok{(LatteSubset, BigMacSubset)}
\KeywordTok{str}\NormalTok{(FinalMerge, }\DataTypeTok{vec.len=}\DecValTok{2}\NormalTok{)}
\end{Highlighting}
\end{Shaded}

\begin{verbatim}
## 'data.frame':    29 obs. of  4 variables:
##  $ country         : chr  "Argentina" "Australia" ...
##  $ city            : chr  "Buenos Aires" "Sydney" ...
##  $ GrandeLatteIndex: num  4.18 4.82 5.65 4.23 3.81 ...
##  $ BigMacIndex     : num  3.82 4.9 ...
\end{verbatim}

\hypertarget{una-pequeuxf1a-introducciuxf3n-a-las-bases-de-datos}{%
\subsection{Una pequeña introducción a las bases de
datos}\label{una-pequeuxf1a-introducciuxf3n-a-las-bases-de-datos}}

\hypertarget{bases-de-datos-relacionales-consultas-y-sql}{%
\subsubsection{Bases de datos relacionales, consultas y
SQL}\label{bases-de-datos-relacionales-consultas-y-sql}}

\begin{enumerate}
\def\labelenumi{\arabic{enumi}.}
\tightlist
\item
  Las bases de datos son diseñadas: esto involucra decidir que variables
  son incluidas y como seran oraganizadas.
\item
  El diseño de bases de datos son implementados en un entorno especifico
  de un software, en el cual implique crear y poblar la tabla de datos,
  proporcionando aceso a los datos.
\item
  Correr el Consultador SQL para volver a extraer de la base de datos un
  dato especifico que se necesite.
\end{enumerate}

\hypertarget{una-introducciuxf3n-a-el-paquete-sqldf}{%
\subsubsection{Una Introducción a el paquete
SQLDF}\label{una-introducciuxf3n-a-el-paquete-sqldf}}

Este paquete no proporciona un consultor SQL forma de utilizarlo es la
siguiente:

\begin{Shaded}
\begin{Highlighting}[]
\KeywordTok{library}\NormalTok{(sqldf)}
\NormalTok{strangeCars <-}\StringTok{ }\KeywordTok{sqldf}\NormalTok{(}\StringTok{"SELECT city , price, country}
\StringTok{                      FROM LatteIndexFrame}
\StringTok{                      WHERE price > 3 AND price < 5"}\NormalTok{) }\CommentTok{# o cambiar Where por GROUP BY}
\NormalTok{strangeCars}
\end{Highlighting}
\end{Shaded}

\begin{verbatim}
##             city price       country
## 1    Mexico City  3.22        Mexico
## 2  San Francisco  3.55 United States
## 3        Detroit  3.55 United States
## 4         London  3.81       England
## 5        Atlanta  3.83 United States
## 6      Hong Kong  3.87         China
## 7       Istanbul  3.92        Turkey
## 8         Lisbon  4.05      Portugal
## 9        Toronto  4.08 United States
## 10  Buenos Aires  4.18     Argentina
## 11     Sao Paolo  4.23        Brazil
## 12      New York  4.30 United States
## 13        Dublin  4.38       Ireland
## 14         Tokyo  4.49         Japan
## 15    Wellington  4.51   New Zealand
## 16         Seoul  4.54   South Korea
## 17        Madrid  4.65         Spain
## 18       Beijing  4.81         China
## 19        Sydney  4.82     Australia
\end{verbatim}

También cuenta con unión entre bases de datos usando el siguiente
programa:

\begin{Shaded}
\begin{Highlighting}[]
\NormalTok{query <-}\StringTok{ "SELECT Lsub.country, Lsub.city, Lsub.GrandeLatteIndex, Msub.BigMacIndex}
\StringTok{FROM LatteSubset AS Lsub INNER JOIN BigMacSubset AS Msub}
\StringTok{ON Lsub.country = Msub.country"}
\NormalTok{IndexFrame <-}\StringTok{ }\KeywordTok{sqldf}\NormalTok{(query)}
\KeywordTok{head}\NormalTok{(IndexFrame)}
\end{Highlighting}
\end{Shaded}

\begin{verbatim}
##         country          city GrandeLatteIndex BigMacIndex
## 1         India     New Delhi             2.79    1.666823
## 2        Mexico   Mexico City             3.22    2.904603
## 3 United States San Francisco             3.55    4.367396
## 4 United States       Detroit             3.55    4.367396
## 5       Britain        London             3.81    4.248183
## 6 United States       Atlanta             3.83    4.367396
\end{verbatim}

\begin{Shaded}
\begin{Highlighting}[]
\KeywordTok{library}\NormalTok{(RSQLite)}
\NormalTok{conn <-}\StringTok{ }\KeywordTok{dbConnect}\NormalTok{(}\KeywordTok{SQLite}\NormalTok{(), }\StringTok{"EconomicIndexDatabase.db"}\NormalTok{)}
\KeywordTok{dbWriteTable}\NormalTok{(conn, }\StringTok{"GrandeLatteTable"}\NormalTok{, LatteSubset)}
\KeywordTok{dbWriteTable}\NormalTok{(conn, }\StringTok{"BigMacTable"}\NormalTok{, BigMacSubset)}
\end{Highlighting}
\end{Shaded}

\begin{Shaded}
\begin{Highlighting}[]
\KeywordTok{dbListTables}\NormalTok{(conn)}
\end{Highlighting}
\end{Shaded}

\begin{verbatim}
## [1] "BigMacTable"      "GrandeLatteTable"
\end{verbatim}

\begin{Shaded}
\begin{Highlighting}[]
\KeywordTok{dbListFields}\NormalTok{(conn, }\StringTok{"GRANDELATTETABLE"}\NormalTok{)}
\end{Highlighting}
\end{Shaded}

\begin{verbatim}
## [1] "country"          "city"             "GrandeLatteIndex"
\end{verbatim}

\begin{Shaded}
\begin{Highlighting}[]
\KeywordTok{dbListFields}\NormalTok{(conn, }\StringTok{"bigmactable"}\NormalTok{)}
\end{Highlighting}
\end{Shaded}

\begin{verbatim}
## [1] "country"     "BigMacIndex"
\end{verbatim}

\begin{Shaded}
\begin{Highlighting}[]
\KeywordTok{dbGetQuery}\NormalTok{(conn, }\StringTok{"SELECT COUNT(*) AS 'GrandeLatteRowCount' FROM GrandeLatteTable"}\NormalTok{)}
\end{Highlighting}
\end{Shaded}

\begin{verbatim}
##   GrandeLatteRowCount
## 1                  29
\end{verbatim}

\begin{Shaded}
\begin{Highlighting}[]
\KeywordTok{dbGetQuery}\NormalTok{(conn, }\StringTok{"SELECT COUNT(*) AS 'BigMacRowCount' FROM BigMacTable"}\NormalTok{)}
\end{Highlighting}
\end{Shaded}

\begin{verbatim}
##   BigMacRowCount
## 1             57
\end{verbatim}

\begin{Shaded}
\begin{Highlighting}[]
\NormalTok{query <-}\StringTok{ "SELECT M.country, M.BigMacIndex FROM BigMacTable AS M}
\StringTok{WHERE M.country IN (SELECT country FROM GrandeLatteTable)"}
\NormalTok{BigMacBoth <-}\StringTok{ }\KeywordTok{dbGetQuery}\NormalTok{(conn, query)}
\end{Highlighting}
\end{Shaded}

\begin{Shaded}
\begin{Highlighting}[]
\NormalTok{query <-}\StringTok{ "SELECT M.country, M.BigMacIndex FROM BigMacTable AS M}
\StringTok{WHERE M.country NOT IN (SELECT country FROM GrandeLatteTable)"}
\NormalTok{BigMacOnly <-}\StringTok{ }\KeywordTok{dbGetQuery}\NormalTok{(conn, query)}
\KeywordTok{dbDisconnect}\NormalTok{(conn)}
\end{Highlighting}
\end{Shaded}

\hypertarget{model-de-regresiuxf3n-lineal}{%
\section{Model de Regresión Lineal}\label{model-de-regresiuxf3n-lineal}}

Los modelos predictivos son modelos matemáticos que nos permiten
predecir algunas variables de interes desde una o mas variables que se
cree se relacionan.

\hypertarget{modelando-los-datos-de-witeside}{%
\subsection{Modelando los datos de
Witeside}\label{modelando-los-datos-de-witeside}}

\begin{Shaded}
\begin{Highlighting}[]
\KeywordTok{library}\NormalTok{(MASS)}
\KeywordTok{summary}\NormalTok{(whiteside)}
\end{Highlighting}
\end{Shaded}

\begin{verbatim}
##     Insul         Temp             Gas       
##  Before:26   Min.   :-0.800   Min.   :1.300  
##  After :30   1st Qu.: 3.050   1st Qu.:3.500  
##              Median : 4.900   Median :3.950  
##              Mean   : 4.875   Mean   :4.071  
##              3rd Qu.: 7.125   3rd Qu.:4.625  
##              Max.   :10.200   Max.   :7.200
\end{verbatim}

\begin{Shaded}
\begin{Highlighting}[]
\KeywordTok{par}\NormalTok{(}\DataTypeTok{mfrow=}\KeywordTok{c}\NormalTok{(}\DecValTok{1}\NormalTok{,}\DecValTok{1}\NormalTok{))}
\NormalTok{x <-}\StringTok{ }\NormalTok{whiteside}\OperatorTok{$}\NormalTok{Temp}
\NormalTok{y <-}\StringTok{ }\NormalTok{whiteside}\OperatorTok{$}\NormalTok{Gas}
\KeywordTok{plot}\NormalTok{(x, y, }\DataTypeTok{xlab =} \StringTok{"Temperatura"}\NormalTok{, }\DataTypeTok{ylab =} \StringTok{"Gas"}\NormalTok{)}
\NormalTok{olsModel <-}\StringTok{ }\KeywordTok{lm}\NormalTok{(y }\OperatorTok{~}\StringTok{ }\NormalTok{x)}
\KeywordTok{abline}\NormalTok{(olsModel, }\DataTypeTok{lty =} \DecValTok{2}\NormalTok{,}\DataTypeTok{lwd =} \DecValTok{3}\NormalTok{)}
\end{Highlighting}
\end{Shaded}

\includegraphics{Exposición_files/figure-latex/unnamed-chunk-28-1.pdf}

\hypertarget{describiendo-lineas-en-el-plano}{%
\subsubsection{Describiendo lineas en el
plano}\label{describiendo-lineas-en-el-plano}}

\begin{Shaded}
\begin{Highlighting}[]
\KeywordTok{plot}\NormalTok{(}\DecValTok{0}\NormalTok{,}\DecValTok{0}\NormalTok{)}
\KeywordTok{abline}\NormalTok{(}\DataTypeTok{a=}\DecValTok{1}\NormalTok{, }\DataTypeTok{b=}\DecValTok{2}\NormalTok{, }\DataTypeTok{lty=}\DecValTok{2}\NormalTok{, }\DataTypeTok{lwd=}\DecValTok{3}\NormalTok{)}
\end{Highlighting}
\end{Shaded}

\includegraphics{Exposición_files/figure-latex/unnamed-chunk-29-1.pdf}

\begin{Shaded}
\begin{Highlighting}[]
\KeywordTok{plot}\NormalTok{(}\DecValTok{0}\NormalTok{,}\DecValTok{0}\NormalTok{)}
\KeywordTok{abline}\NormalTok{(}\DataTypeTok{a=}\DecValTok{0}\NormalTok{,}\DataTypeTok{b=}\DecValTok{0}\NormalTok{, }\DataTypeTok{lty=}\DecValTok{2}\NormalTok{, }\DataTypeTok{lwd=}\DecValTok{3}\NormalTok{)}
\end{Highlighting}
\end{Shaded}

\includegraphics{Exposición_files/figure-latex/unnamed-chunk-29-2.pdf}

\begin{Shaded}
\begin{Highlighting}[]
\KeywordTok{plot}\NormalTok{(}\DecValTok{0}\NormalTok{,}\DecValTok{0}\NormalTok{)}
\KeywordTok{abline}\NormalTok{(}\DataTypeTok{v=}\DecValTok{0}\NormalTok{, }\DataTypeTok{lty=}\DecValTok{2}\NormalTok{, }\DataTypeTok{lwd=}\DecValTok{3}\NormalTok{)}
\end{Highlighting}
\end{Shaded}

\includegraphics{Exposición_files/figure-latex/unnamed-chunk-29-3.pdf}

\begin{Shaded}
\begin{Highlighting}[]
\NormalTok{linearModelA <-}\StringTok{ }\KeywordTok{lm}\NormalTok{(Gas }\OperatorTok{~}\StringTok{ }\NormalTok{Temp, }\DataTypeTok{data =}\NormalTok{ whiteside)}
\KeywordTok{names}\NormalTok{(linearModelA)}
\end{Highlighting}
\end{Shaded}

\begin{verbatim}
##  [1] "coefficients"  "residuals"     "effects"       "rank"         
##  [5] "fitted.values" "assign"        "qr"            "df.residual"  
##  [9] "xlevels"       "call"          "terms"         "model"
\end{verbatim}

\begin{Shaded}
\begin{Highlighting}[]
\KeywordTok{plot}\NormalTok{(}\DecValTok{0}\NormalTok{,}\DecValTok{0}\NormalTok{)}
\KeywordTok{abline}\NormalTok{(linearModelA, }\DataTypeTok{lty =} \DecValTok{2}\NormalTok{, }\DataTypeTok{lwd =} \DecValTok{2}\NormalTok{)}
\end{Highlighting}
\end{Shaded}

\includegraphics{Exposición_files/figure-latex/unnamed-chunk-30-1.pdf}

\begin{Shaded}
\begin{Highlighting}[]
\NormalTok{linearModelA}\OperatorTok{$}\NormalTok{coefficients}
\end{Highlighting}
\end{Shaded}

\begin{verbatim}
## (Intercept)        Temp 
##   5.4861933  -0.2902082
\end{verbatim}


\end{document}
